\tongjisetup{
  %=========
  % 中文信息
  %=========
  ctitle={基于一种细长圆杆的多维弹塑性分析},
  cheadingtitle={基于一种细长圆杆的多维弹塑性分析},   
  cauthor={唐霖辉},  
  studentnumber={2410486},
  ccategories={工学},
  cmajorfirst={土木工程},
  cmajorsecond={结构工程},
  cdepartment={土木工程学院},
  csupervisor={何敏娟 教授}, 
  cresearchfield={大跨木结构},
  cassosupervisor={任晓丹 教授},
  %=========
  % 英文信息
  %=========
  etitle={Multi-dimensional elastoplastic analysis based on an elongated round rod}, 
  eauthor={Linhui Tang},
  ecategories={Gong Xue},
  emajorfirst={Civil Engineering},
  emajorsecond={Structural Engineering},
  edepartment={School of Civil Engineering},    
  esupervisor={Prof. Minjuan He},
  eassosupervisor={Prof. Xiaodan Ren},
  eresearchfield={Large-span timber construction},
  }
  %=========
  % 中英文摘要和关键字
  %=========
\begin{cabstract}  
弹塑性模型与实际应用问题相对应。近年来,弹塑性模型的应用有两种主要的模式:一是在抽象和简化的基础上建立简单的模型,针对模型求得考虑材料弹塑性的解析解;二是建立实际工程问题的有限元分析模型,结合弹塑性模型的数值算法,求得工程问题的数值解。本文将基于这两种模式,对某细长圆杆进行多维弹塑性分析。同时对该细长圆杆进行有限元模拟,根据计算结果讨论两种方法的区别和联系,并研究理想弹塑性材料与弹塑性硬化材料的区别。
\end{cabstract}

\ckeywords{弹塑性力学, 多维弹塑性理论, 屈服函数, 有限元分析}

\begin{eabstract}
Elastic-plastic modeling corresponds to practical application problems. In recent years, there are two main modes for the application of elastic-plastic modeling: one is to establish a simple model on the basis of abstraction and simplification, for which the analytical solution considering the elastic-plasticity of the material is obtained; the other is to establish a finite element analytical model of the actual engineering problem, and then combine with numerical algorithms of elastic-plastic modeling to obtain the numerical solution of the engineering problem. In this paper, a multi-dimensional elastic-plastic analysis of an elongated circular rod is carried out based on these two models. At the same time, finite element simulation is carried out on this slender circular rod, and the differences and connections between the two methods are discussed based on the calculation results, and the differences between ideal elastic-plastic materials and elastic-plastic hardened materials are studied.
\end{eabstract}

\ekeywords{elastoplasticity, multidimensional elastoplasticity theory, yield function, finite element analysis}