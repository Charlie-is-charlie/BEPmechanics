\chapter{D-P~模型材料杆的有限元分析}
\label{cha:DP}
其他条件不变,将材料考虑为~D-P~硬化材料,假设内摩擦角为~{$10^\circ$},剪胀角为~{$10^\circ$},将模型材料进行调整,再进行有限元分析。Mises~应力云图见图7.1,S33~应力云图见图7.2,S23~应力云图见图7.3。绘制圆杆跨中截面正应力与剪应力分布云图如图7.4、图7.5所示。

导出~D-P~模型材料条件下,圆杆跨中截面正应力与剪应力的径向分布结果,并与理想弹塑性材料、线性硬化材料条件下的有限元分析结果进行比较,绘制散点图如图7.6、图7.7所示。

由图7.6、图7.7可知,总体而言,三种材料条件下,圆杆跨中截面正应力和剪应力的分布符合相同趋势。D-P~模型下,截面正应力分布曲线接近理想弹塑性材料条件下的正应力分布曲线,截面剪应力分布曲线与线性硬化材料条件下的剪应力分布曲线较为吻合。