\chapter{各向同性线性硬化材料杆的理论求解}
\label{cha:hardened}
其他条件不变,将材料考虑为各向同性线性硬化材料,硬化模量为~{$k$}。式~(\ref{eq17})~转变为:
\begin{equation}\label{eq28}
    \dot s_{ij} = 2G(e_{ij}-\dfrac{3}{2}\dfrac{s_{ij}}{\sigma_{y}^2(1+\dfrac{K}{3G})}s_{kl} \dot e_{kl})
\end{equation}

因为硬化模量~{$k$}~为常数,故式~(\ref{eq28})~仍是关于~{$s_{ij}$}~的封闭常微分方程,存在解析解。将~{$f_y$}~用~{$f_y\sqrt{1+\dfrac{K}{3G}}$}~代替,带入式~(\ref{eq28}),可得各向同性线性硬化材料条件下,圆杆轴向应力与剪应力沿杆截面分布的解析表达式:
\begin{subequations}\label{eq29}
    \begin{numcases} 
        ~\sigma = 2f_{y} \dfrac{\exp(\dfrac{3Gr\theta}{f_{y} L}\sqrt{\dfrac{G}{3G+K}})}{\exp(\dfrac{6Gr\theta}{f_{y} L}\sqrt{\dfrac{G}{3G+K}}) + 1} \sqrt{1+\dfrac{K}{3G}} \label{eq29a} \\
        \tau = \dfrac{f_{y}}{\sqrt{3}}\dfrac{\exp(\dfrac{6Gr\theta}{f_{y} L}\sqrt{\dfrac{G}{3G+K}}) - 1}{\exp(\dfrac{6Gr\theta}{f_{y} L}\sqrt{\dfrac{G}{3G+K}}) + 1} \sqrt{1+\dfrac{K}{3G}}  \label{eq29b}
    \end{numcases}
\end{subequations}