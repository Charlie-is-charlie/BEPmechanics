\chapter{理想弹塑性材料杆的理论求解}
\label{cha:ideal_theory}
本章节将推导圆杆变形之后,轴向应力与剪应力沿杆截面分布的解析表达式。

首先轴向拉伸细杆,使其全长全截面轴向应力均达到~{$f_y$},建立柱坐标系,此时圆杆内任一横截面上距离轴线~{$x$}距离为~{$r$}的点的应力状态为:
\begin{equation}\label{eq4}
    \sigma_{ij} = \begin{bmatrix}
        0 & 0 & 0 \\
        0 & f_y & 0 \\
        0 & 0 & 0
        \end{bmatrix}
\end{equation} 

偏应力张量:
\begin{equation}\label{eq5}
    s_{ij} = \begin{bmatrix}
        -\frac{1}{3}f_y & 0 & 0 \\
        0 & \frac{2}{3}f_y & 0 \\
        0 & 0 & -\frac{1}{3}f_y
        \end{bmatrix}
\end{equation} 

代入~Mises~屈服准则,求得:
\begin{equation}\label{eq6}
    q = \frac{f_y}{\sqrt{3}}
\end{equation} 

考虑将杆的两个端截面相对旋转一个小角度~{$\theta$},则单位扭转角为:
\begin{equation}\label{eq7}
    \alpha  = \frac{\theta }{L}
\end{equation} 

根据几何关系,柱坐标系下的微应变张量的各分量为:


此时圆杆的位移场: