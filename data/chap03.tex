\chapter{理想弹塑性材料杆的理论求解}
\label{cha:ideal_theory}
本章节将推导圆杆变形之后,轴向应力与剪应力沿杆截面分布的解析表达式。参考的教材除《工程弹塑性力学引论》外,还借鉴了《弹性与塑性力学》\cite{txysxlx}。

首先轴向拉伸细杆,使其全长全截面轴向应力均达到~{$f_y$},建立柱坐标系,此时圆杆内任一横截面上距离轴线~{$x$}~距离为~{$r$}~的点的应力状态为:
\begin{equation}\label{eq4}
    \sigma_{ij} = \begin{bmatrix}
        0 & 0 & 0 \\
        0 & f_y & 0 \\
        0 & 0 & 0
        \end{bmatrix}
\end{equation} 

偏应力张量:
\begin{equation}\label{eq5}
    s_{ij} = \begin{bmatrix}
        -\frac{1}{3}f_y & 0 & 0 \\
        0 & \frac{2}{3}f_y & 0 \\
        0 & 0 & -\frac{1}{3}f_y
        \end{bmatrix}
\end{equation} 

代入~Mises~屈服准则,求得:
\begin{equation}\label{eq6}
    q = \frac{f_y}{\sqrt{3}}
\end{equation} 

考虑将杆的两个端截面相对旋转一个小角度~{$\theta$},则单位扭转角为:
\begin{equation}\label{eq7}
    \alpha  = \frac{\theta }{L}
\end{equation} 

根据几何关系,柱坐标系下的微应变张量的各分量为:
\begin{equation}\label{eq8}
    \left\{
        \begin{array}{l}
            \varepsilon_{rr} = \dfrac{\partial u_r}{\partial r};~ \varepsilon_{\theta\theta} = \dfrac{\partial u_\theta}{\partial \theta};~ \varepsilon_xx = \dfrac{\partial u_x}{\partial x}\vspace{1ex}\\
            \varepsilon_{r\theta} = \dfrac{1}{2}(\dfrac{1}{r}\dfrac{\partial u_r}{\partial \theta}+\dfrac{\partial u_\theta}{\partial r}-\dfrac{u_\theta}{r})\vspace{1ex}\\
            \varepsilon_{\theta x} = \varepsilon_{x \theta}=\dfrac{1}{2}(\dfrac{1}{r}\dfrac{\partial u_x}{\partial \theta}+\dfrac{\partial u_\theta}{\partial x})\vspace{1ex}\\
            \varepsilon_{xr} = \varepsilon_{rx}=\dfrac{1}{2}(\dfrac{\partial u_x}{\partial r}+\dfrac{\partial u_r}{\partial x})\\
        \end{array}
    \right.
\end{equation} 

此时圆杆的位移场:
\begin{equation}\label{eq9}
    \left\{
        \begin{array}{l}
            u_x = \dfrac{f_y}{E}x \vspace{1ex}\\
            u_\theta = axr\\
        \end{array}
    \right.
\end{equation} 

则有:
\begin{equation}\label{eq10}
    \varepsilon_{\theta x} =\dfrac{1}{2}(\dfrac{1}{r}\dfrac{\partial u_x}{\partial \theta}+\dfrac{\partial u_\theta}{\partial x})=\dfrac{1}{2}ar=\dfrac{\theta r}{2L}
\end{equation} 

此时圆杆上各点的应变状态为:
\begin{equation}\label{eq11}
    \varepsilon_{ij} = \begin{bmatrix}
        \varepsilon_{rr} & 0 & 0 \vspace{1ex}\\
        0 & \varepsilon_{xx} & \dfrac{\theta r}{2L} \vspace{1ex}\\
        0 & \dfrac{\theta r}{2L} & \varepsilon_{\theta \theta}
        \end{bmatrix}
\end{equation} 

由问题叙述已知,扭转过程中杆长不变,则~{$\varepsilon_{rr}$}、{$\varepsilon_{xx}$}、~{$\varepsilon_{\theta \theta}$}~在旋转过程中保持不变。因此,可得应变率张量:
\begin{equation}\label{eq12}
    \dot \varepsilon_{ij} = \begin{bmatrix}
        0 & 0 & 0 \vspace{1ex}\\
        0 & 0 & \dfrac{\dot\theta r}{2L}\\
        0 & \dfrac{\dot\theta r}{2L} & 0
    \end{bmatrix}
\end{equation} 

求得偏应变率张量:
\begin{equation}\label{eq13}
    \dot e_{ij} = \dot \varepsilon_{ij} - \dfrac{1}{3}\dot \varepsilon_{ij} \delta_{ij}  = \begin{bmatrix}
        0 & 0 & 0 \vspace{1ex}\\
        0 & 0 & \dfrac{\dot\theta r}{2L}\\
        0 & \dfrac{\dot\theta r}{2L} & 0
        \end{bmatrix}
\end{equation} 

此外,写出此时圆杆各点应力状态:
\begin{equation}\label{eq14}
    \sigma_{ij} = \begin{bmatrix}
        0 & 0 & 0 \\
        0 & \sigma_{xx} & \tau_{x \theta}\\
        0 & \tau_{\theta x} & 0
        \end{bmatrix}
\end{equation} 

则有偏应力张量:
\begin{equation}\label{eq15}
    s_{ij} = \begin{bmatrix}
        -\dfrac{1}{3}\sigma_{xx} & 0 & 0 \vspace{1ex}\\
        0 & \dfrac{2}{3}\sigma_{xx} & \tau_{x \theta}\vspace{1ex}\\
        0 & \tau_{\theta x} & -\dfrac{1}{3}\sigma_{xx}
        \end{bmatrix}
\end{equation} 

相对应的偏应力率张量:
\begin{equation}\label{eq16}
    \dot s_{ij} = \begin{bmatrix}
        -\dfrac{1}{3}\dot \sigma_{xx} & 0 & 0 \vspace{1ex}\\
        0 & \dfrac{2}{3}\dot \sigma_{xx} & \dot \tau_{x \theta}\vspace{1ex}\\
        0 & \dot \tau_{\theta x} & \dfrac{1}{3}\dot \sigma_{xx}
        \end{bmatrix}
\end{equation} 

考虑相关塑性流动法则下,基于~Mises~屈服准则建立的多维弹塑性理论模型。偏应力率与偏应变率的解析求解关系:
\begin{equation}\label{eq17}
    \dot s_{ij} = 2G(\delta_{ik}\delta_{jl}-\dfrac{3}{2}\dfrac{s_{ij}s_{kl}}{\sigma_{y}^2}) \dot e_{kl} \\
    = 2G(e_{ij}-\dfrac{3}{2}\dfrac{s_{ij}}{\sigma_{y}^2}s_{kl} \dot e_{kl})
\end{equation}
其中,{$G=\frac{E}{2(1+2\nu)}$}

因为此时圆杆内各点屈服,由~Mises~屈服函数可得:
\begin{equation}\label{eq18}
    \sigma_y = \sqrt{\dfrac{3}{2} s_{ij} s_{ij}} = \sqrt{\dfrac{3}{2}(\dfrac{2}{3}\sigma_{xx}^2 + 2\tau_{x \theta}^2)} = f_y
\end{equation}

整理得圆杆截面各点正应力~{$\sigma_{xx}$}~与剪应力~{$\tau_{x \theta}$}~满足屈服条件:
\begin{equation}\label{eq19}
    \sigma_{xx}^2 + 3\tau_{x \theta}^2 = f_{y}^2
\end{equation}

将式~(\ref{eq13})、(\ref{eq15})、(\ref{eq16})~代入式~(\ref{eq17}),得偏应变率~{$\dot s_{22}$}、{$\dot s_{23}$}:
\begin{subequations}\label{eq20}
    \begin{numcases} 
        ~\dot s_{22} = \dfrac{2}{3}\dot\sigma_{xx} = 2G(0-\dfrac{\sigma_{xx}}{f_{y}^2} \dfrac{\tau_{x \theta}}{L} r\dot \theta)  \\
        \dot s_{23} = \dot\tau_{x \theta} = 2G(\dfrac{\dot\theta r}{2L}-\dfrac{3}{2} \dfrac{\tau_{x \theta}^2}{f_{y}^2} \dfrac{r}{L} \dot\theta)
    \end{numcases}
\end{subequations}

将~{$\sigma_{xx}$}~表示为~{$\sigma$},将~{$\tau_{x \theta}$}~表示为~{$\tau$},整理可得:
\begin{subequations}\label{eq21}
    \begin{numcases} 
        ~\dfrac{1}{\sigma} d\sigma = -\dfrac{3Gr}{f_{y}^2 L} \tau d \theta \label{eq21a}  \\
        \dfrac{2 \sqrt{3}f_{y}}{f_{y}^2 - 3\tau^2} d \tau = \dfrac{2 \sqrt{3} Gr}{f_{y} L} d \theta \label{eq21b}
    \end{numcases}
\end{subequations}

对式~\eqref{eq21b}~两边积分,得:
\begin{equation}\label{eq22}
    \ln(f_y + \sqrt{3}\tau) - \ln(f_y - \sqrt{3}\tau) + C_2 = \dfrac{2 \sqrt{3} Gr}{f_{y} L} \theta
\end{equation}
将~{$\theta = 0$},{$\tau = 0$}~代入,解得~{$C_2 = 0$}。反代回式~\eqref{eq22}~解出~{$\tau$}~的表达式:
\begin{equation}\label{eq23}
    \tau = \dfrac{f_{y}}{\sqrt{3}}\dfrac{\exp(\dfrac{2 \sqrt{3} Gr}{f_{y} L} \theta) - 1}{\exp(\dfrac{2 \sqrt{3} Gr}{f_{y} L} \theta) + 1}
\end{equation}

将式~\eqref{eq23}~代入式~\eqref{eq21a},并对两边积分,得:
\begin{equation}\label{eq24}
    \ln(\sigma) + C_1 = \dfrac{\sqrt{3}Gr}{f_{y}L} \theta -\ln(\exp(\dfrac{2 \sqrt{3} Gr}{f_{y} L} \theta)+1)
\end{equation}
同样的,将~{$\theta = 0$},{$\sigma = f_y$}~代入,解得~{$C_1 = -\ln(2f_y)$}。反代回式~\eqref{eq24}~解出~{$\sigma$}~的表达式:
\begin{equation}\label{eq25}
    \sigma = 2f_{y} \dfrac{\exp(\dfrac{\sqrt{3} Gr}{f_{y} L} \theta)}{\exp(\dfrac{2 \sqrt{3} Gr}{f_{y} L} \theta) + 1}
\end{equation}

将解得的~{$\sigma$}~和~{$\tau$}~代入式~\eqref{eq19},有:
\begin{equation}\label{eq26}
    \sigma^2 + 3\tau^2 =  f_{y}^2 \dfrac{4exp(2k)+exp(4k)-2exp(2k)+1}{(exp(2k)+1)^2} = f_{y}^2
\end{equation}
其中,{$k = \dfrac{\sqrt{3} Gr}{f_{y} L} \theta$}。\vspace{1ex}

因此,验证推导结果满足~Mises~屈服条件。综上,理想弹塑性材料条件下,圆杆轴向应力与剪应力沿杆截面分布的解析表达式为:
\begin{subequations}\label{eq27}
    \begin{numcases} 
        ~\sigma = 2f_{y} \dfrac{\exp(\dfrac{\sqrt{3} Gr}{f_{y} L} \theta)}{\exp(\dfrac{2 \sqrt{3} Gr}{f_{y} L} \theta) + 1} \label{eq27a}  \\
        \tau = \dfrac{f_{y}}{\sqrt{3}}\dfrac{\exp(\dfrac{2 \sqrt{3} Gr}{f_{y} L} \theta) - 1}{\exp(\dfrac{2 \sqrt{3} Gr}{f_{y} L} \theta) + 1} \label{eq27b}
    \end{numcases}
\end{subequations}