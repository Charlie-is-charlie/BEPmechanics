\chapter{理想弹塑性材料杆的理论求解}
\label{cha:ideal_theory}
本章节将推导圆杆变形之后,轴向应力与剪应力沿杆截面分布的解析表达式。

首先轴向拉伸细杆,使其全长全截面轴向应力均达到~{$f_y$},建立柱坐标系,此时圆杆内任一横截面上距离轴线~{$x$}距离为~{$r$}的点的应力状态为:
\begin{equation}\label{eq4}
    \sigma_{ij} = \begin{bmatrix}
        0 & 0 & 0 \\
        0 & f_y & 0 \\
        0 & 0 & 0
        \end{bmatrix}
\end{equation} 

偏应力张量:
\begin{equation}\label{eq5}
    s_{ij} = \begin{bmatrix}
        -\frac{1}{3}f_y & 0 & 0 \\
        0 & \frac{2}{3}f_y & 0 \\
        0 & 0 & -\frac{1}{3}f_y
        \end{bmatrix}
\end{equation} 

代入~Mises~屈服准则,求得:
\begin{equation}\label{eq6}
    q = \frac{f_y}{\sqrt{3}}
\end{equation} 

考虑将杆的两个端截面相对旋转一个小角度~{$\theta$},则单位扭转角为:
\begin{equation}\label{eq7}
    \alpha  = \frac{\theta }{L}
\end{equation} 

根据几何关系,柱坐标系下的微应变张量的各分量为:
\begin{equation}\label{eq8}
    \left\{
        \begin{array}{l}
            \varepsilon_rr = \dfrac{\partial u_r}{\partial r}; \varepsilon_{\theta\theta} = \dfrac{\partial u_\theta}{\partial \theta}; \varepsilon_xx = \dfrac{\partial u_x}{\partial x}\vspace{1ex}\\
            \varepsilon_{r\theta} = \dfrac{1}{2}(\dfrac{1}{r}\dfrac{\partial u_r}{\partial \theta}+\dfrac{\partial u_\theta}{\partial r}-\dfrac{u_\theta}{r})\vspace{1ex}\\
            \varepsilon_{\theta x} = \varepsilon_{x \theta}=\dfrac{1}{2}(\dfrac{1}{r}\dfrac{\partial u_x}{\partial \theta}+\dfrac{\partial u_\theta}{\partial x})\vspace{1ex}\\
            \varepsilon_{xr} = \varepsilon_{rx}=\dfrac{1}{2}(\dfrac{\partial u_x}{\partial r}+\dfrac{\partial u_r}{\partial x})\\
        \end{array}
    \right.
\end{equation} 

此时圆杆的位移场:
\begin{equation}\label{eq9}
    \left\{
        \begin{array}{l}
            u_x = \dfrac{f_y}{E}x \vspace{1ex}\\
            u_\theta = axr\\
        \end{array}
    \right.
\end{equation} 

则有:
\begin{equation}\label{eq10}
    \varepsilon_{\theta x} =\dfrac{1}{2}(\dfrac{1}{r}\dfrac{\partial u_x}{\partial \theta}+\dfrac{\partial u_\theta}{\partial x})=\dfrac{1}{2}ar=\dfrac{\theta r}{2L}
\end{equation} 

此时圆杆上各点的应变状态为:
\begin{equation}\label{eq11}
    \varepsilon_{ij} = \begin{bmatrix}
        \varepsilon_{rr} & 0 & 0 \vspace{1ex}\\
        0 & \varepsilon_{xx} & \dfrac{\theta r}{2L} \vspace{1ex}\\
        0 & \dfrac{\theta r}{2L} & \varepsilon_{\theta \theta}
        \end{bmatrix}
\end{equation} 

由问题叙述已知,扭转过程中杆长不变,则~{$\varepsilon_{rr}$}、{$\varepsilon_{xx}$}、~{$\varepsilon_{\theta \theta}$}~在旋转过程中保持不变。因此,可得应变率张量:
\begin{equation}\label{eq12}
    \dot e_{ij} = \dot \varepsilon_{ij} - \dfrac{1}{3}\dot \varepsilon_{ij} \delta_{ij}  = \begin{bmatrix}
        0 & 0 & 0 \vspace{1ex}\\
        0 & 0 & \dfrac{\dot\theta r}{2L}\\
        0 & \dfrac{\dot\theta r}{2L} & 0
        \end{bmatrix}
\end{equation} 

此外,写出此时圆杆各点应力状态:
\begin{equation}\label{eq13}
    \sigma_{ij} = \begin{bmatrix}
        0 & 0 & 0 \\
        0 & \sigma_{xx} & \tau_{x \theta}\\
        0 & \tau_{\theta x} & 0
        \end{bmatrix}
\end{equation} 

则有偏应力张量:
\begin{equation}\label{eq14}
    s_{ij} = \begin{bmatrix}
        -\dfrac{1}{3}\sigma_{xx} & 0 & 0 \vspace{1ex}\\
        0 & \dfrac{2}{3}\sigma_{xx} & \tau_{x \theta}\vspace{1ex}\\
        0 & \tau_{\theta x} & -\dfrac{1}{3}\sigma_{xx}
        \end{bmatrix}
\end{equation} 

相对应的偏应力率张量:
\begin{equation}\label{eq15}
    \dot s_{ij} = \begin{bmatrix}
        -\dfrac{1}{3}\dot \sigma_{xx} & 0 & 0 \vspace{1ex}\\
        0 & \dfrac{2}{3}\dot \sigma_{xx} & \dot \tau_{x \theta}\vspace{1ex}\\
        0 & \dot \tau_{\theta x} & \dfrac{1}{3}\dot \sigma_{xx}
        \end{bmatrix}
\end{equation} 

考虑相关塑性流动法则下,基于~Mises~屈服准则建立的多维弹塑性理论模型。偏应力率与偏应变率的解析求解关系:

因为此时圆杆内各点屈服,由~Mises~屈服函数可得:

整理得圆杆截面各点正应力 与剪应力 满足屈服条件:




