\chapter{问题描述}
\label{cha:Description}
给定一根圆形截面细杆,已知: 

1.~长度为~{$L$},半径为~{$R$}; 

2.~构成材料为理想弹塑性材料,弹性模量~{$E$},泊松比~{$\nu$},屈服强度为~{$f_y$},服从~Mises~屈服准则。 

加载方式如下: 

1.~首先轴向拉伸细杆,使其全长全截面轴向应力均达到~{$f_y$}; 

2.~然后在保持杆长不变的情况下,杆的两个端截面相对旋转一个小角度~{$\theta$},假定扭转变形沿着杆的轴向均匀分布,同时杆的任意横截面在旋转过程中均保持平面。 

问: 
1.如何不经过计算,即判断杆的轴向应力在旋转变形过程中是变大还是变小? 

2.推导上述变形之后,轴向应力与剪应力沿杆截面分布的解析表达式。

3.其他条件不变,将材料考虑为各向同性线性硬化材料,硬化模量为~{$K$},推导上述问题的控制方程,并讨论此时是否可以得到问题的解析解。

4.自拟一组参数(建议采用钢材的材料参数,钢材强度等级自定),进行上述问题的有限元分析(软件不限),比较理论结果与计算结果,讨论二者的区别与联系。 

5.其他条件不变,将材料考虑为各向同性线性硬化材料,硬化模量取为,,再进行有限元分析,根据计算结果讨论理想弹塑性材料与弹塑性硬化材料的区别。 

6.如果改为Drucker-Prager塑性模型,结果将会怎样? 
