\chapter{轴向应力的变化趋势}
\setlength{\footskip}{15.61334pt}
\label{cha:trend}
本章节将不经计算,通过基本公式判断杆的轴向应力在旋转变形过程中是变大还是变小。

经历轴向拉伸至全长全截面屈服后再旋转小角度过程中,Mises~屈服准则为:
\begin{equation}\label{eq1}
    f(\sigma_{ij}) = \sqrt{J_2}-q = 0
\end{equation} 
其中,{$J_2$}~为应力的一个不变量,{$q$}~为屈服参数。

在主应力空间的柱坐标系中,式~(\ref{eq1})可表示为:
\begin{equation}\label{eq2}
    \sqrt{\frac{1}{6}[(\sigma_{rr}-\sigma_{\theta \theta})^2+(\sigma_{\theta \theta}-\sigma_{xx})^2+(\sigma_{xx}-\sigma_{rr})^2]+\sigma_{r\theta}^2+\sigma_{x\theta}^2+\sigma_{xr}^2}-q = 0
\end{equation}
先将圆杆轴向拉伸至~{$f_y$},然后在保持杆长不变的情况下,杆的两个端截面相对旋转一个小角度~{$\theta $}。旋转过程中,圆杆任一横截面上各点仅存在正应力~{$\sigma_{xx}$}~与剪应力~{$\sigma_{x\theta}$}~不为~0。代入~Mises~屈服函数式~(\ref{eq2}),在旋转过程中,总有下式成立:
\begin{equation}\label{eq3}
    \sqrt{\frac{1}{3}\sigma_{xx}^2+\sigma_{x\theta}^2} = q
\end{equation}

因为在旋转过程中剪应力~{$\sigma_{x\theta}$}~不断增大,而理想弹塑性材料条件下~{$q$}~为常数,故可以判断截面正应力~{$\sigma_{xx}$}~在旋转变形过程中不断变小。
