\chapter{总结}
\label{cha:conclusion}
在分析细长圆杆拉伸扭转变形时,理想弹塑性模型假设材料在屈服后进入完全塑性阶段,不考虑硬化效应。此模型下,杆件一旦屈服,材料的应力保持恒定,随扭矩增大,材料将直接进入塑性变形区域,直至失效。该模型适用于对材料性质要求不严格的初步分析,但无法准确描述材料在实际扭转中的延展性和强度变化。

相比之下,考虑硬化的弹塑性模型则引入了材料屈服后继续硬化的效应,假设材料在塑性变形中,屈服强度随着变形的增加而逐步增强。对于钢材,硬化模型能够更真实地模拟其在受扭过程中的表现,尤其是在高扭矩作用下,材料能够承受更大的变形和扭矩,具有更高的韧性和耐久性。因此,硬化模型能提供更为准确的预测,特别是对于需要较大变形的工程结构\cite{Lemaitre1990}。

在材料屈服准则的选择上,von Mises~模型适用于~Q345~钢这类金属材料,其屈服标准主要基于剪应力,即扭转引起的应力,能够较好地描述金属材料的塑性行为。与之相比,D-P~模型考虑了体积效应,通常用于脆性材料如混凝土和岩石,适合在压缩状态下描述材料屈服行为。对于钢材的受扭分析,von Mises~准则更为适用,因为钢材的屈服行为主要受剪应力影响,不需要考虑体积变化效应\cite{Barrett2014}。